% Este archivo corresponde a los paquete y combinaciónes de colores necesarias para dejarlo con los colores azules de la UCM
% Plantilla realizada por Carlos Palacios junio de 2015
\usepackage{amsmath}
\usepackage{amsfonts}
\usepackage{amssymb}
\usetheme{CambridgeUS}

\usepackage{xcolor}

\setbeamercolor{normal text}		{fg=black,bg=white} %color del texto normal
\setbeamercolor{alerted text}		{fg=blue}
\setbeamercolor{example text}		{fg=black}
\setbeamercolor{background canvas}  {fg=pink, bg=white}%diapositiva de fondo blanco
\setbeamercolor{background}         {fg=green, bg=yellow}
\setbeamercolor{frametitle}			{fg=blue!60!black,bg=gray!10}%color del texto del título y su fondo
\setbeamercolor{title}				{fg=yellow!10!white,bg=blue!60!black}%color del texto del título y su fondo, solo para la lámina de título
\setbeamercolor{palette primary}    {fg=blue!60!black, bg=gray!30!white}%colores de las distintas paletas
\setbeamercolor{palette secondary}  {fg=blue!70!black, bg=gray!20!white}%colores de las distintas paletas
\setbeamercolor{palette tertiary}   {fg=white, bg=blue!60!black}%colores de las distintas paletas
\setbeamercolor{block title}{bg=blue!60!black,fg=white}%color del título de un bloque
%\setbeamercolor{block body}{fg=black,bg=gray!10}

%\setbeamercolor{author in head/foot}{bg=blue}
%\setbeamercolor{section in head/foot}{fg=yellow}

\setbeamercolor{author in title}{fg=blue}
\setbeamercovered{transparent=10}

\usepackage{hyperref}
