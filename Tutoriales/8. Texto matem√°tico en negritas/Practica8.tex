\documentclass[12pt]{article}
%%%%%%%%%%%%%%%%%%%%%%%%%%%%%
% Preambulo
%
\usepackage[T1]{fontenc}
\usepackage[utf8]{inputenc}
\usepackage[spanish,es-tabla]{babel}% agregamos es-tabla
\parindent=0cm %modificar tamaño de sangria 
\usepackage{amsmath}
\usepackage{amssymb,amsfonts,latexsym,cancel}
\usepackage{graphicx}
\usepackage{epstopdf}
\usepackage{float}
\usepackage{subfigure}
\usepackage{array}
\usepackage{longtable}
\newcolumntype{E}{>{$}c<{$}}
\setcounter{MaxMatrixCols}{40}
%%%%%%%%%%%%%%%%%%%%%%%%%%%%%%%%%%%%
%% Paquetes o configuración nueva %%
%%%%%%%%%%%%%%%%%%%%%%%%%%%%%%%%%%%%
\usepackage{bm}
%
\begin{document}
\title{Practica 8.\\ Texto matemático en negritas}
\author{Héctor Misael}
\date{}
\maketitle
\tableofcontents

\section{Negritas}

\textbf{Esto} esta en negritas $ \boldsymbol{a+b} $ %Texto matemático en negrita

\[
a+b
\]

\[
\boldsymbol{
a+b
}
\]

\[
\bm{a+b} 
\] %Para esto se necesita el paqute {bm}

\begin{table}[H] %H es para que el cuadro se coloque de forma personalizada
\begin{tabular}{E|E|E}
\text{Texto normal} & \text{Usando boldsymbol} & \text{Usando bm} \\
x+1 & \boldsymbol{x+1} & \bm{x+1} \\
x^{2} & \boldsymbol{x^{2}} & \bm{x^{2}} \\
\sum x+2 &  \boldsymbol{\sum x+2} & \bm{\sum x+2} \\
\tan x &  \boldsymbol{\tan x} & \bm{\tan x} \\
\end{tabular}
\end{table}


\end{document}