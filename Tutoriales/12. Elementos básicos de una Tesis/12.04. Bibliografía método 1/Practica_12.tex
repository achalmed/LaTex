\documentclass[12pt,oneside]{book}
%
\usepackage[lmargin=3.81cm,rmargin=2.54cm,top=2.54cm,
bottom=2.54cm]{geometry}
\usepackage[T1]{fontenc}
\usepackage[utf8]{inputenc}
\usepackage[spanish,es-tabla]{babel}
\parindent=0cm
\usepackage{amsmath}
\usepackage{amssymb,amsfonts,latexsym,cancel}
\usepackage{array}
\usepackage{bm}
\usepackage{float}
\usepackage{fancyhdr}
\usepackage{graphicx}
\usepackage{epstopdf}
\usepackage[colorlinks = true,
            linkcolor = blue,
            citecolor = black,
            urlcolor = blue]{hyperref}
\usepackage{longtable}
\setcounter{MaxMatrixCols}{40}
\usepackage{multicol}
\usepackage{subfigure}
\usepackage{titling}
\usepackage{titlesec}
\newcolumntype{E}{>{$}c<{$}}

%
\makeatletter
\let\ps@plain\ps@empty
\makeatother
%
\pagestyle{fancy}
\fancyfoot{}
\fancyhead{}
\fancyhead[C]{\scriptsize \leftmark}
\fancyfoot[C]{\scriptsize \rightmark}
\fancyhead[R]{\thepage}
\renewcommand{\headrulewidth}{0.8pt}
\renewcommand{\footrulewidth}{0.5pt}
\begin{document}
\frontmatter
\begin{center}
\vspace*{\baselineskip}

{
\bf\fontsize{19}{0}{\selectfont{UNIVERSIDAD DE GUADALAJARA}}\\[0.5cm]
\fontsize{11}{0}{CENTRO UNIVERSITARIO DE CIENCIAS EXACTAS E INGENIERÍAS}
}

\vspace*{0.5\baselineskip}

{
\bf\fontsize{14}{0}{\selectfont{NOMBRE DE LA DIVISIÓN}}
}

\vspace*{0.5\baselineskip}

{
\bf\fontsize{11}{0}{\selectfont{NOMBRE DEL DEPARTAMENTO AL QUE PERTENECEN}}
}

\vspace*{\baselineskip}
\includegraphics[scale=0.3]{figuras/udg}
\vspace*{3\baselineskip}

{
\bf\fontsize{13}{0}{\selectfont{TÍTULO DE TESIS}}
}

\vspace*{4.5\baselineskip}

{
\hfill\bf\fontsize{14}{0}{\selectfont{Director de tesis: Nombre}}\\[0.2cm]
\hfill\bf\fontsize{14}{0}{\selectfont{Tesista: Nombre}}
}

\vfill

Guadalajara, Jalisco, \hfill Abril del 2017

\thispagestyle{empty}

\end{center}
\include{codigofuente/agradecimientos}
\renewcommand{\contentsname}{Contenido}
\tableofcontents

\mainmatter 
\chapter{Cálculo diferencial}
\section{Funciones}
Se presenta una introducción a los siguientes temas El producto de dos binomios conjugados es igual al cuadrado de el primer término menos el cuadrado del segundo término.
Se presenta una introducción a los siguientes temas El producto de dos binomios conjugados es igual al cuadrado de el primer término menos el cuadrado del segundo término.
Se presenta una introducción a los siguientes temas El producto de dos binomios conjugados es igual al cuadrado de el primer término menos el cuadrado del segundo término.
Se presenta una introducción a los siguientes temas El producto de dos binomios conjugados es igual al cuadrado de el primer término menos el cuadrado del segundo término. 

\section{Límites}
Se presenta una introducción a los siguientes temas El producto de dos binomios conjugados es igual al cuadrado de el primer término menos el cuadrado del segundo término.
Se presenta una introducción a los siguientes temas El producto de dos binomios conjugados es igual al cuadrado de el primer término menos el cuadrado del segundo término.
Se presenta una introducción a los siguientes temas El producto de dos binomios conjugados es igual al cuadrado de el primer término menos el cuadrado del segundo término.
Se presenta una introducción a los siguientes temas El producto de dos binomios conjugados es igual al cuadrado de el primer término menos el cuadrado del segundo término.

\section{Continuidad}
Se presenta una introducción a los siguientes temas El producto de dos binomios conjugados es igual al cuadrado de el primer término menos el cuadrado del segundo término.
Se presenta una introducción a los siguientes temas El producto de dos binomios conjugados es igual al cuadrado de el primer término menos el cuadrado del segundo término.
Se presenta una introducción a los siguientes temas El producto de dos binomios conjugados es igual al cuadrado de el primer término menos el cuadrado del segundo término.
Se presenta una introducción a los siguientes temas El producto de dos binomios conjugados es igual al cuadrado de el primer término menos el cuadrado del segundo término.

\section{Derivadas}
Se presenta una introducción a los siguientes temas El producto de dos binomios conjugados es igual al cuadrado de el primer término menos el cuadrado del segundo término.
Se presenta una introducción a los siguientes temas El producto de dos binomios conjugados es igual al cuadrado de el primer término menos el cuadrado del segundo término.
Se presenta una introducción a los siguientes temas El producto de dos binomios conjugados es igual al cuadrado de el primer término menos el cuadrado del segundo término.
Se presenta una introducción a los siguientes temas El producto de dos binomios conjugados es igual al cuadrado de el primer término menos el cuadrado del segundo término.

\section{Regla de la cadena}
Se presenta una introducción a los siguientes temas El producto de dos binomios conjugados es igual al cuadrado de el primer término menos el cuadrado del segundo término.
Se presenta una introducción a los siguientes temas El producto de dos binomios conjugados es igual al cuadrado de el primer término menos el cuadrado del segundo término.
Se presenta una introducción a los siguientes temas El producto de dos binomios conjugados es igual al cuadrado de el primer término menos el cuadrado del segundo término.
Se presenta una introducción a los siguientes temas El producto de dos binomios conjugados es igual al cuadrado de el primer término menos el cuadrado del segundo término.

\chapter{Álgebra}
\section{Factorización}
Se presenta una introducción a los siguientes temas El producto de dos binomios conjugados es igual al cuadrado de el primer término menos el cuadrado del segundo término.
Se presenta una introducción a los siguientes temas El producto de dos binomios conjugados es igual al cuadrado de el primer término menos el cuadrado del segundo término.
Se presenta una introducción a los siguientes temas El producto de dos binomios conjugados es igual al cuadrado de el primer término menos el cuadrado del segundo término.
Se presenta una introducción a los siguientes temas El producto de dos binomios conjugados es igual al cuadrado de el primer término menos el cuadrado del segundo término.

\section{Binomios conjugados}
Se presenta una introducción a los siguientes temas El producto de dos binomios conjugados es igual al cuadrado de el primer término menos el cuadrado del segundo término.
Se presenta una introducción a los siguientes temas El producto de dos binomios conjugados es igual al cuadrado de el primer término menos el cuadrado del segundo término.
Se presenta una introducción a los siguientes temas El producto de dos binomios conjugados es igual al cuadrado de el primer término menos el cuadrado del segundo término.
Se presenta una introducción a los siguientes temas El producto de dos binomios conjugados es igual al cuadrado de el primer término menos el cuadrado del segundo término.

\section{Exponentes}
Se presenta una introducción a los siguientes temas El producto de dos binomios conjugados es igual al cuadrado de el primer término menos el cuadrado del segundo término.
Se presenta una introducción a los siguientes temas El producto de dos binomios conjugados es igual al cuadrado de el primer término menos el cuadrado del segundo término.
Se presenta una introducción a los siguientes temas El producto de dos binomios conjugados es igual al cuadrado de el primer término menos el cuadrado del segundo término.
Se presenta una introducción a los siguientes temas El producto de dos binomios conjugados es igual al cuadrado de el primer término menos el cuadrado del segundo término.

\section{Raíces}
Se presenta una introducción a los siguientes temas El producto de dos binomios conjugados es igual al cuadrado de el primer término menos el cuadrado del segundo término.
Se presenta una introducción a los siguientes temas El producto de dos binomios conjugados es igual al cuadrado de el primer término menos el cuadrado del segundo término.
Se presenta una introducción a los siguientes temas El producto de dos binomios conjugados es igual al cuadrado de el primer término menos el cuadrado del segundo término.
Se presenta una introducción a los siguientes temas El producto de dos binomios conjugados es igual al cuadrado de el primer término menos el cuadrado del segundo término.

\section{División sintética}
Se presenta una introducción a los siguientes temas El producto de dos binomios conjugados es igual al cuadrado de el primer término menos el cuadrado del segundo término.
Se presenta una introducción a los siguientes temas El producto de dos binomios conjugados es igual al cuadrado de el primer término menos el cuadrado del segundo término.
Se presenta una introducción a los siguientes temas El producto de dos binomios conjugados es igual al cuadrado de el primer término menos el cuadrado del segundo término.
Se presenta una introducción a los siguientes temas El producto de dos binomios conjugados es igual al cuadrado de el primer término menos el cuadrado del segundo término.

\chapter{Geometría modificada}
\section{Triángulos}
Se presenta una introducción a los siguientes temas El producto de dos binomios conjugados es igual al cuadrado de el primer término menos el cuadrado del segundo término.
Se presenta una introducción a los siguientes temas El producto de dos binomios conjugados es igual al cuadrado de el primer término menos el cuadrado del segundo término.
Se presenta una introducción a los siguientes temas El producto de dos binomios conjugados es igual al cuadrado de el primer término menos el cuadrado del segundo término.
Se presenta una introducción a los siguientes temas El producto de dos binomios conjugados es igual al cuadrado de el primer término menos el cuadrado del segundo término.

\section{Linea recta}
Se presenta una introducción a los siguientes temas El producto de dos binomios conjugados es igual al cuadrado de el primer término menos el cuadrado del segundo término.
Se presenta una introducción a los siguientes temas El producto de dos binomios conjugados es igual al cuadrado de el primer término menos el cuadrado del segundo término.
Se presenta una introducción a los siguientes temas El producto de dos binomios conjugados es igual al cuadrado de el primer término menos el cuadrado del segundo término.
Se presenta una introducción a los siguientes temas El producto de dos binomios conjugados es igual al cuadrado de el primer término menos el cuadrado del segundo término.

Se presenta una introducción a los siguientes temas El producto de dos binomios conjugados es igual al cuadrado de el primer término menos el cuadrado del segundo término.
Se presenta una introducción a los siguientes temas El producto de dos binomios conjugados es igual al cuadrado de el primer término menos el cuadrado del segundo término.
Se presenta una introducción a los siguientes temas El producto de dos binomios conjugados es igual al cuadrado de el primer término menos el cuadrado del segundo término.
Se presenta una introducción a los siguientes temas El producto de dos binomios conjugados es igual al cuadrado de el primer término menos el cuadrado del segundo término.
Se presenta una introducción a los siguientes temas El producto de dos binomios conjugados es igual al cuadrado de el primer término menos el cuadrado del segundo término.
Se presenta una introducción a los siguientes temas El producto de dos binomios conjugados es igual al cuadrado de el primer término menos el cuadrado del segundo término.
Se presenta una introducción a los siguientes temas El producto de dos binomios conjugados es igual al cuadrado de el primer término menos el cuadrado del segundo término.
Se presenta una introducción a los siguientes temas El producto de dos binomios conjugados es igual al cuadrado de el primer término menos el cuadrado del segundo término.
Se presenta una introducción a los siguientes temas El producto de dos binomios conjugados es igual al cuadrado de el primer término menos el cuadrado del segundo término.
Se presenta una introducción a los siguientes temas El producto de dos binomios conjugados es igual al cuadrado de el primer término menos el cuadrado del segundo término.
Se presenta una introducción a los siguientes temas El producto de dos binomios conjugados es igual al cuadrado de el primer término menos el cuadrado del segundo término.
Se presenta una introducción a los siguientes temas El producto de dos binomios conjugados es igual al cuadrado de el primer término menos el cuadrado del segundo término.

\chapter{Álgebra lineal}
\section{Matrices}
Se presenta una introducción a los siguientes temas El producto de dos binomios conjugados es igual al cuadrado de el primer término menos el cuadrado del segundo término.
Se presenta una introducción a los siguientes temas El producto de dos binomios conjugados es igual al cuadrado de el primer término menos el cuadrado del segundo término.
Se presenta una introducción a los siguientes temas El producto de dos binomios conjugados es igual al cuadrado de el primer término menos el cuadrado del segundo término.
Se presenta una introducción a los siguientes temas El producto de dos binomios conjugados es igual al cuadrado de el primer término menos el cuadrado del segundo término.

\section{Sistemas de ecuaciones}
Se presenta una introducción a los siguientes temas El producto de dos binomios conjugados es igual al cuadrado de el primer término menos el cuadrado del segundo término.
Se presenta una introducción a los siguientes temas El producto de dos binomios conjugados es igual al cuadrado de el primer término menos el cuadrado del segundo término.
Se presenta una introducción a los siguientes temas El producto de dos binomios conjugados es igual al cuadrado de el primer término menos el cuadrado del segundo término.
Se presenta una introducción a los siguientes temas El producto de dos binomios conjugados es igual al cuadrado de el primer término menos el cuadrado del segundo término.


\begin{thebibliography}{1}
\bibitem{Thomas} Thomas, G.B. and Finney, R.L. and Weir, M.D, \emph{Cálculo: una variable}, 1998. Pearson Educación. 
\bibitem{Larson} Larson, R.E.H, \emph{Cálculo}, 1995. McGraw-Hill.
\bibitem{Stewart} Stewart, J, \emph{Calculo: Trascendentes Tempranas}, 2002. Cengage Learning Editores.
\end{thebibliography}

\end{document}