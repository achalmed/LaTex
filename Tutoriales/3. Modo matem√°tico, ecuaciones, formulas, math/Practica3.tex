\documentclass[12pt]{article}
%%%%%%%%%%%%%%%%%%%%%%%%%%%%%%
% Preambulo
%%
\usepackage[T1]{fontenc}
\usepackage[utf8]{inputenc}
\usepackage[spanish]{babel}
\parindent = 0cm
%%
\usepackage{amsmath}
\usepackage{amssymb,amsfonts,latexsym,cancel}
\begin{document}
\title{Practica 3.\\ Texto en modo matemático }
\author{Héctor Misael}
\date{}
\maketitle
\tableofcontents

\section{Texto en modo matemático}
Básicamente existen dos formas de colocar texto en modo matemático en \LaTeX, una de ellas es colocar formulas junto al texto, y la otra es colocarla de forma independiente.
\subsection{Fórmulas junto al texto}
\textbf{Cuadrado de un binomio}. Sea. $ (a+b)^{2} = (a+b)(a+b) $ donde $a$ y $b$ representan números algebraicos cualesquiera, positivos o negativos. Por lo tanto $ (a+b)^{2} = a^{2} +2ab +b^{2} $.
\subsection{Fórmulas independientes}
\textbf{Cuadrado de un binomio}. Sea. 
\[
 (a+b)^{2} = (a+b)(a+b) 
\]
 donde $a$ y $b$ representan números algebraicos cualesquiera, positivos o negativos. Por lo tanto 
\[ 
  (a+b)^{2} = a^{2} +2ab +b^{2} .
\]

\begin{equation*}
 (a+b)^{2} = a^{2} +2ab +b^{2} .
\end{equation*}
\subsection{Numeración de fórmulas y referencias}
El comando para numerar una formula es 
\begin{equation}\label{binomio}
 (a+b)^{2} = a^{2} +2ab +b^{2} .
\end{equation}

\begin{equation}\label{binomio2}
 (x+y)^{2} = x^{2} +2xy +y^{2} .
\end{equation}
Ahora usando la ecuación \ref{binomio} y \ref{binomio2}. \\
Ahora usando la ecuación \eqref{binomio} y \eqref{binomio2}. 

\end{document}