\documentclass[12pt]{article}
%%%%%%%%%%%%%%%%%%%%%%%%%%%%%
% Preambulo
\usepackage[T1]{fontenc}
\usepackage[utf8]{inputenc}
\usepackage[spanish,es-tabla]{babel}
\parindent=0cm %modificar tamaño de sangria 
\usepackage{amsmath}
\usepackage{amssymb,amsfonts,latexsym,cancel}
\usepackage{graphicx}
\usepackage{epstopdf}
\usepackage{float}
\usepackage{subfigure}
\usepackage{array}
\usepackage{longtable}
\newcolumntype{E}{>{$}c<{$}}
\setcounter{MaxMatrixCols}{40}
\usepackage{bm}
%%%%%%%%%%%%%%%%%%%%%%%%%%%%%%%%%%%%%%
%% Paquetes o configuración nueva 
%---->%%%%%%%%%%%%%%%%%%%%%%%%%%%%%%%%
\usepackage[lmargin=2cm,rmargin=2cm,top=2.5cm,
bottom=2cm]{geometry}
\usepackage{fancyhdr}
\pagestyle{fancy}
\fancyhead{} % Eleminamos definiciones previas
\fancyhead[C]{Titulo del articulo}
\fancyhead[R]{\includegraphics[scale=0.03]{figuras/infinito}}
\fancyfoot{}
\fancyfoot[R]{\thepage}
\fancyfoot[L]{Héctor Misael}
\renewcommand{\headrulewidth}{0.9pt}
\renewcommand{\footrulewidth}{0.5pt}
%---->%%%%%%%%%%%%%%%%%%%%%%%%%%%%%%%%
\begin{document}

\begin{titlepage}
\begin{center}
\vspace*{2\baselineskip}
\hrule height 3pt
\vspace*{0.5\baselineskip}
{\Huge \textbf{Nombre de institución}}\\[0.1cm]
{\large \textbf{CENTRO UNIVERSITARIO AL CUAL PERTENECEN}}
\vspace*{0.5\baselineskip}
\hrule
\vspace*{7\baselineskip}
\includegraphics[scale=0.15]{figuras/cuadro}
\vspace*{2\baselineskip}

\textbf{ÁLGEBRA, CÁLCULO DIFERENCIAL \\
ejemplos y aplicaciones}
\vfill
HÉCTOR MISAEL BACILIO \\
\today
\end{center}
\end{titlepage}

%\renewcommand{\contentsname}{Contenido}
\tableofcontents
\thispagestyle{empty}
\newpage
\setcounter{page}{1}

\section{Introducción}
\section{Bases}
\subsection{Productos notables}
\section{Cálculo diferencial}
\subsection{Razón de cambio y límites}
\subsection{Límites laterales y límites al infinito}
\subsection{Continuidad}

\newpage
%\renewcommand{\refname}{Bibliografía}
\begin{thebibliography}{99}
\bibitem{Nombre1} Articulo o Libro 1, Autor 1, Año de referencia 1. 
\bibitem{Nombre2} Articulo o Libro 2, Autor 2, Año de referencia 2. 
\bibitem{Nombre3} Articulo o Libro 3, Autor 3, Año de referencia 3. 
\bibitem{Nombre4} Articulo o Libro 4, Autor 4, Año de referencia 4. 
\end{thebibliography}

\end{document}