\documentclass[12pt]{article}

%%%%%%%%%%%%%%%%%%%%%%%%%%%%%
% Preámbulo
%%
\usepackage[T1]{fontenc}
\usepackage[utf8]{inputenc}
\usepackage[spanish]{babel}
%%

%\parindent = 1cm Este comando configura las sangrías de los capítulos, parrafos y títulos%

\begin{document}

\title{Practica 2. \\ Texto normal, párrafos y alineación.}
\author{Edison Achalma}
\date{march 12 2021}
\maketitle
%Después de tener sección y subsecciones ya se puede generar tabla de contenidos%
\tableofcontents
%\tableofcontents después de insertar este comando solo se compila unas tres veces para que el tabla de cotenidos aparesca de forma automática%

\section{Tipos y tamaños de fuente}

Iniciamos con la segunda práctica de este tutorial, colocando el título autor y día. Esta es la primera sección donde conoceremos algunos tipos de letra, así como diferentes tamaños de letras que podemos usar.

\subsection{Tipos de letra}

\begin{itemize}
\item Tipo de letra \textbf{negrita}
\item Tipo de letra \textit{itálica}
\item Tipo de letra \textrm{romana}
\item Tipo de letra \textsf{sans serif}
\item Tipo de letra \texttt{mono espaciada}
\item Tipo de letra \textsl{inclinada}
\item Tipo de letra \textsc{versalitas}
\end{itemize}

\newpage %Este comando hace que el párrafo siguiente pase al siguiente página%

\subsection{Tamaños de letra}

\begin{itemize}
\item {\tiny Tamaño} de letra
\item {\scriptsize Tamaño} de letra
\item {\footnotesize Tamaño} de letra
\item {\small Tamaño} de letra
\item {\normalsize Tamaño} de letra
\item {\large Tamaño} de letra
\item {\Large Tamaño} de letra
\item {\LARGE Tamaño} de letra
\item {\huge Tamaño} de letra
\item {\huge Tamaño} de letra
\end{itemize}

\section{Párrafos, sangría y saltos de línea.}

\subsection{Sangría}

\noindent %Este comando quita la sangría% Esta es la primera sección donde conoceremos algunos tipos de letra, así como diferentes tamaños de letras que podemos usar.
Y ahora queremos también decidir modificar la sangría en el documento, aquí sí iniciamos con sangría.

\noindent Pero esta otra línea queremos que NO tenga sangría

Pero esta otra línea queremos que SÍ tenga sangría

\subsection{Salto de línea y nueva página}

Ahora queremos también decidir modificar la sangría en el documento, aquí sí iniciamos con sangría. \\ 
% \\ Este comando hace saltar al siguiente párrafo, pero el párrafo comienza sin sangría.%
Pero esta otra línea que no tenga sangría.
\par 
%\par Este comando hace saltar al siguiente párrafo, además el párrafo comienza con sangría.%
Esta línea queremos que sí tenga sangría y además un espacio mayor. \\[0.5cm] 
%\\[0.5cm] Este comando hace saltar al siguiente párrafo, pero el párrafo comienza sin sangría. Además agrega un espacio de Xcm o Ymm entre los dos párrafos%
Pero esta otra línea queremos que no tenga sangría.

\subsection{Alineación de párrafos}
%Aquí están los comandos para alinear párrafos, textos, figuras o tablas.%

\begin{flushleft}
Texto a la izquierda
\end{flushleft}

\begin{center}
Texto centrado
\end{center}

\begin{flushright}
Texto a la derecha
\end{flushright}

\subsection{Comillas y puntos suspensivos}

Este es un ejemplo: “para las comillas”, 'simples' y las comillas “dobles”. \\
Y también los puntos suspensivos \dots %\dots Este comando genera los puntos suspensivos%

\subsection{Espaciado horizontal}
\noindent Inicio \, fin \\[0.2cm]
%\, Este comandno genera un espacio entre palabras%
Inicio \quad fin \\[0.2cm]
%\quad Este comando genera un espacio un poco mas entre letras%
Inicio \qquad fin \\[0.2cm]
Inicio \hspace{3cm} fin \\[0.2cm]
%\hspace{3cm} Comando para espaciado entre letras de forma personalizada%
Inicio \hfill fin \\[0.2cm]
%\hfill Comando que separa las palabras de un extremo a otro de forma horizontal%
\subsection{Líneas de relleno}
\noindent Inicio \hfill fin \\[0.5cm]
Inicio \hrulefill fin \\[0.5cm]
%\hrulefill comando que rellena con línea entre palabras de un extremo a otro%
Inicio \dotfill fin \\[0.5cm]
%\hrulefill comando que rellena con puntos entre palabras de un extremo a otro%

% despues de tner las secciones y subsecciones se procedió a generar el tablade contenidos%

%Este formato también puede servir para un libro, para esto solo se debe cambiar el tipo de documento al principio del docuemto y luego antes de la sección se agrega un capítulo con el siguiente comando \chapter{Capítulo 1} por ejemplo.%

\end{document}