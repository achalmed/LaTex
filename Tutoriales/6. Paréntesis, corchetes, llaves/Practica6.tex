\documentclass[12pt]{article}
%%%%%%%%%%%%%%%%%%%%%%%%%%%%%
% Preambulo
%
\usepackage[T1]{fontenc}
\usepackage[utf8]{inputenc}
\usepackage[spanish,es-tabla]{babel}% agregamos es-tabla
\parindent=0cm %modificar tamaño de sangria 
\usepackage{amsmath}
\usepackage{amssymb,amsfonts,latexsym,cancel}
\usepackage{graphicx}
\usepackage{epstopdf}
\usepackage{float}
\usepackage{subfigure}
%
\usepackage{array}
\usepackage{longtable}
\newcolumntype{E}{>{$}c<{$}}
%%%%%%%%%%%%%%%%%%%%%%%%%%%%%
\begin{document}
\title{Practica 6.\\ Paréntesis, corchetes y llaves}
\author{Edison Achalma}
\date{}
\maketitle
\tableofcontents

\section{Símbolos de agrupación}

\[
w + ( \frac{d}{b+c} ) = (a+b+c) 
\]

\[
w + \left( \frac{d}{b+c} \right)   = (a+b+c) 
\]

\[
w + \left[ \frac{d}{b+c} \right]   = (a+b+c) 
\]

\[
w + \left[ \frac{d}{b+c} \right.   = (a+b+c) 
\]

\[
\left\lbrace 
\frac{a+b}{c+d} 
\right. 
\]

\begin{equation}
\left( 
\frac{a+b}{c+d}
\right\rbrace 
\end{equation}

\begin{equation}
\left\lbrace 
\frac{a+b}{c+d}
\right\rbrace 
\end{equation}

\begin{equation}
\left( 
\frac{a+b}{c+d}
\right)^{5} 
\end{equation}

%Se puso un delimitador, en ves de un punto una barra.
\begin{equation}
\left. 
\frac{dy}{dx}
\right|_{x=1} = x+1
\end{equation}

\section{Tamaño manual}

\[
( \sum_{i=1}^{n} )
\]

\[
\bigl(
\sum_{i=1}^{n} 
\bigr)
\]

\[
\Bigl(
\sum_{i=1}^{n} 
\Bigr)
\]

\[
\biggl(
\sum_{i=1}^{n} 
\biggr)
\]

\[
\Biggl(
\sum_{i=1}^{n} 
\Biggr)
\]


\[
\Bigl(
(a+b)^2 + c
\Bigr)
\]


\end{document}