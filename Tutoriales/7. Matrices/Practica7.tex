\documentclass[12pt]{article}
%%%%%%%%%%%%%%%%%%%%%%%%%%%%%
% Preambulo
%
\usepackage[T1]{fontenc}
\usepackage[utf8]{inputenc}
\usepackage[spanish,es-tabla]{babel}% agregamos es-tabla
\parindent=0cm %modificar tamaño de sangria 
\usepackage{amsmath}
\usepackage{amssymb,amsfonts,latexsym,cancel}
\usepackage{graphicx}
\usepackage{epstopdf}
\usepackage{float}
\usepackage{subfigure}
\usepackage{array}
\usepackage{longtable}
\newcolumntype{E}{>{$}c<{$}}
%%%%%%%%%%%%%%%Paquete nuevo
\setcounter{MaxMatrixCols}{40}
%%%%%%%%%%%%%%%%%%%%%%%%%%%%%
\begin{document}
\title{Practica 7\\ Matrices}
\author{Héctor Misael}
\date{}
\maketitle
\tableofcontents

\section{Matrices usando el entorno array}

\[
\begin{array}{lc}
0 & 1 \\
2 & 3 
\end{array}
\]

\[
\left( 
\begin{array}{lc}
0 & 1 \\
2 & 3 
\end{array}
\right) 
\]

\[
\left( 
\begin{array}{lcl}
0 & x & 1 \\
1 & x+1 & 2 \\
10 & x+y+1 & 3 
\end{array}
\right) 
\]
\newpage

\section{Matrices con entornos predefinidos}

\[
\begin{matrix}
0 & 1 & 2 \\
0 & 1 & 2 \\
0 & 1 & 2 \\
\end{matrix}
\]

\[
\begin{pmatrix}
0 & 1 & 2 \\
0 & 1 & 2 \\
0 & 1 & 2 \\
\end{pmatrix}
\]

\[
\begin{bmatrix}
0 & 1 & 2 \\
0 & 1 & 2 \\
0 & 1 & 2 \\
\end{bmatrix}
\]

\[
\begin{vmatrix}
0 & 1 & 2 \\
0 & 1 & 2 \\
0 & 1 & 2 \\
\end{vmatrix}
\]

\[
\begin{Bmatrix}
0 & 1 & 2 \\
0 & 1 & 2 \\
0 & 1 & 2 \\
\end{Bmatrix}
\]

\[
\begin{Vmatrix}
0 & 1 & 2 \\
0 & 1 & 2 \\
0 & 1 & 2 \\
\end{Vmatrix}
\]

\section{Máximo número de columnas}

\[
\begin{pmatrix}
 1 & 2 & 3 & 4 & 5 & 6 & 7 & 8 & 9 & 10 & 11 & 12 & 13 \\
 1 & 2 & 3 & 4 & 5 & 6 & 7 & 8 & 9 & 10 & 11 & 12 & 13
\end{pmatrix}
\]

\end{document}